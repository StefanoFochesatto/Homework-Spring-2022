
%%%%%%%%%%%%%%%%%%%%%%%%%%%%%%%%%%%%%%%%%%%%%%%%%%%%%%%%%%%%%%%%%%%%%%%%%%%%%%%%%%%%%%%
%%%%%%%%%%%%%%%%%%%%%%%%%%%%%%%%%%%%%%%%%%%%%%%%%%%%%%%%%%%%%%%%%%%%%%%%%%%%%%%%%%%%%%%
% 
% This top part of the document is called the 'preamble'.  Modify it with caution!
%
% The real document starts below where it says 'The main document starts here'.

\documentclass[12pt]{article}

\usepackage{amssymb,amsmath,amsthm}
\usepackage[top=1in, bottom=1in, left=1.25in, right=1.25in]{geometry}
\usepackage{fancyhdr}
\usepackage{enumerate}
\usepackage{listings}
\usepackage{graphicx}
\usepackage{float}
% Comment the following line to use TeX's default font of Computer Modern.
\usepackage{times,txfonts}



\makeatletter
\renewcommand*\env@matrix[1][*\c@MaxMatrixCols c]{%
  \hskip -\arraycolsep
  \let\@ifnextchar\new@ifnextchar
  \array{#1}}
\makeatother

\newtheoremstyle{homework}% name of the style to be used
  {18pt}% measure of space to leave above the theorem. E.g.: 3pt
  {12pt}% measure of space to leave below the theorem. E.g.: 3pt
  {}% name of font to use in the body of the theorem
  {}% measure of space to indent
  {\bfseries}% name of head font
  {:}% punctuation between head and body
  {2ex}% space after theorem head; " " = normal interword space
  {}% Manually specify head
\theoremstyle{homework} 

% Set up an Exercise environment and a Solution label.
\newtheorem*{exercisecore}{Exercise \@currentlabel}
\newenvironment{exercise}[1]
{\def\@currentlabel{#1}\exercisecore}
{\endexercisecore}

\newcommand{\localhead}[1]{\par\smallskip\noindent\textbf{#1}\nobreak\\}%
\newcommand\solution{\localhead{Solution:}}

%%%%%%%%%%%%%%%%%%%%%%%%%%%%%%%%%%%%%%%%%%%%%%%%%%%%%%%%%%%%%%%%%%%%%%%%
%
% Stuff for getting the name/document date/title across the header
\makeatletter
\RequirePackage{fancyhdr}
\pagestyle{fancy}
\fancyfoot[C]{\ifnum \value{page} > 1\relax\thepage\fi}
\fancyhead[L]{\ifx\@doclabel\@empty\else\@doclabel\fi}
\fancyhead[C]{\ifx\@docdate\@empty\else\@docdate\fi}
\fancyhead[R]{\ifx\@docauthor\@empty\else\@docauthor\fi}
\headheight 15pt

\def\doclabel#1{\gdef\@doclabel{#1}}
\doclabel{Use {\tt\textbackslash doclabel\{MY LABEL\}}.}
\def\docdate#1{\gdef\@docdate{#1}}
\docdate{Use {\tt\textbackslash docdate\{MY DATE\}}.}
\def\docauthor#1{\gdef\@docauthor{#1}}
\docauthor{Use {\tt\textbackslash docauthor\{MY NAME\}}.}
\makeatother

% Shortcuts for blackboard bold number sets (reals, integers, etc.)
\newcommand{\Reals}{\ensuremath{\mathbb R}}
\newcommand{\Nats}{\ensuremath{\mathbb N}}
\newcommand{\Ints}{\ensuremath{\mathbb Z}}
\newcommand{\Rats}{\ensuremath{\mathbb Q}}
\newcommand{\Cplx}{\ensuremath{\mathbb C}}
%% Some equivalents that some people may prefer.
\let\RR\Reals
\let\NN\Nats
\let\II\Ints
\let\CC\Cplx

%%%%%%%%%%%%%%%%%%%%%%%%%%%%%%%%%%%%%%%%%%%%%%%%%%%%%%%%%%%%%%%%%%%%%%%%%%%%%%%%%%%%%%%
%%%%%%%%%%%%%%%%%%%%%%%%%%%%%%%%%%%%%%%%%%%%%%%%%%%%%%%%%%%%%%%%%%%%%%%%%%%%%%%%%%%%%%%
% 
% The main document start here.

% The following commands set up the material that appears in the header.
\doclabel{Math 410: Homework 4}
\docauthor{Stefano Fochesatto}
\docdate{\today}

\begin{document}



\section*{Section 2.2}

\begin{exercise}{7} Decide whether each of the following sequences converges, and if so, find its limit.\\
    \begin{enumerate}
        \item[d.] $z_n = \dfrac{n(2+i)}{n+1}$\\
        \solution Consider the sequence, after we divide both terms in fraction by $n$, 
        \begin{align*}
            z_n &= \dfrac{n(2+i)}{n+1},\\
            &= \dfrac{\frac{n(2+i)}{n}}{\frac{n+1}{n}},\\
            &= \dfrac{(2+i)}{\frac{n+1}{n}}.
        \end{align*}
        This gives us that the limit as $n \to \infty$ is 
        \begin{equation*}
            \lim_{n \to \infty} z_n = \dfrac{(2+i)}{\frac{n+1}{n}} = 2 + i. 
        \end{equation*}
        \vspace{.15in}
        
        


        \item[e.] $z_n = \left(\dfrac{1 - i}{4}\right)^n$\\
        \solution Consider the moduli of the sequence, 
        \begin{align*}
         |z_n|&= \left|\dfrac{1 - i}{4}\right|^n,\\
         &= \left(\dfrac{|1 - i|}{|4|}\right)^n,\\
         &= \left(\dfrac{\sqrt{2}}{4}\right)^n, \\
         &= 0.
        \end{align*}
        Note that if a sequence $|z_n|$ converges to zero it follows that $z_n$ must also converge to zero. 
        Let $\epsilon >0$ and note that $||z_n| - 0| < \epsilon$ for some $N \in \mathbb{N}$ where $n \geq N$. Note the following, 
        \begin{equation*}
          ||z_n| - 0| = ||z_n|| = |z_n| = |z_n - 0| < \epsilon. 
        \end{equation*}
    \end{enumerate}
\end{exercise}
\vspace{.15in}



\begin{exercise}{11} Find each of the following limits,
  \begin{enumerate}
    \item[d] $\lim_{z \to i} \dfrac{z^2 + i}{z^4 - 1}$\\
    \solution  Note that the limit is undefined at $z = i$, since $(i)^4 - 1 = 1 - 1 = 0$. For $z \neq i$
    consider the following, 
    \begin{align*}
      \lim_{z \to i} \dfrac{z^2 + i}{z^4 - 1} &= \lim_{z \to i} \dfrac{z^2 + i}{(z^2 + 1)(z^2 - 1)},\\
      &= \lim_{z \to i} \dfrac{1}{z^2 - 1}, \\
      &= \dfrac{1}{-2}. 
    \end{align*}

    \vspace{.15in}



    \item[f] $\lim_{z \to 1 + 2i}|z^2 - 1|$\\
    \solution Simply substituting the limit value we get, 
    \begin{align*}
      \lim_{z \to 1 + 2i}|z^2 - 1| &= |(1 + 2i)^2 - 1|,\\
      &=|-4 + 4i|,\\
      &=\sqrt{32},\\
      &=4\sqrt{2}.
    \end{align*}
  \end{enumerate}
\end{exercise}
\vspace{1in}






\begin{exercise}{25} Find each of the following limits involving infinity,\\
  \begin{enumerate}
    \item[a]$\lim_{z \to 2i} \dfrac{z^2 + 9}{2z^2 + 8}$\\
    \solution Simply plugging in the value of the limit we see that the, 
    denominator approaches 0 while the numerator stays constant therefore the 
    limit of the sequence approaches infinity. 
    \begin{align*}
      \lim_{z \to 2i} \dfrac{z^2 + 9}{2z^2 + 8} &=  \dfrac{(2i)^2 + 9}{2(2i)^2 + 8}\\
      &= \dfrac{-4 + 9}{0}\\
      &= \dfrac{5}{0} = \infty. 
    \end{align*}
    \vspace{.15in}
    
  
    \item[b]$\lim_{z \to \infty} \dfrac{3z^2 - 2z}{z^2 - iz + 8}$\\
    \solution  Consider the following factorization, 
    \begin{align*}
      \lim_{z \to \infty} \dfrac{3z^2 - 2z}{z^2 - iz + 8} &= \lim_{z \to \infty} \dfrac{z^2(3 - \frac{2}{z})}{z^2(1 - \frac{i}{z} + \frac{8}{z^2})},\\
      &=\lim_{z \to \infty} \dfrac{3 - \frac{2}{z}}{1 - \frac{i}{z} + \frac{8}{z^2}}.
    \end{align*}
    Applying Theorem 1 we can break apart the limit into the denominator and the numerator, 
    \begin{equation*}
      \lim_{z \to \infty} 3 - \frac{2}{z} = 3 - 0 = 3,
    \end{equation*}
    \begin{equation*}
      \lim_{z \to \infty} 1 - \frac{i}{z} + \frac{8}{z^2} = 1 - 0 + 0 = 1.
    \end{equation*}
    Therefore the final limit is, 
    \begin{equation*}
      \lim_{z \to \infty} \dfrac{3z^2 - 2z}{z^2 - iz + 8} = \dfrac{3}{1} = 3. 
    \end{equation*}
  \end{enumerate}
  
\end{exercise}




\section*{Section 2.3}
\begin{exercise}{7} Use rules (5)-(9) to find the derivatives fo the following function.
  \begin{enumerate}
    \item[b] $f(z) = (z^2 - 3i)^{-6}$ 
    \solution Applying the power rule and chain rule we get the following, 
    \begin{align*}
      f'(z) &= (-6)(z^2 - 3i)^{-7} 2z,\\
      &= (-12)z(z^2 - 3i)^{-7}.
    \end{align*}
    \vspace{.15in}
    
    
    \item[d] $f(z) = \dfrac{(z + 2)^3}{(z^2 + iz + 1)^4}$
    \solution  Applying the quotient rule, power rule, and chain rule we get the following, 
    \begin{align*}
      f'(z) &= \dfrac{ ((z^2 + iz + 1)^4)(3(z + 2)^2) - ((z + 2)^3)(4(2z + i)(z^2 + iz + 1)^3)}{  ((z^2 + iz + 1)^4)^2},\\
            &= \dfrac{(z^2 + iz + 1)(3(z + 2)^2) - (z + 2)^3(4(2z + i))}{(z^2 + iz + 1)^5},\\
            &= \dfrac{ (z + 2)^2 (3(z^2 + iz + 1) - 4(z + 2)(2z + i))}{(z^2 + iz + 1)^5},\\
            &= \dfrac{ (z + 2)^2 (3z^2 + 3iz + 3 - 8z^2 - 4iz - 16z - 8i)}{(z^2 + iz + 1)^5},\\
            &= \dfrac{ (z + 2)^2 (-5z^2 -iz + 3 - 16z - 8i)}{(z^2 + iz + 1)^5}.
    \end{align*}
  \end{enumerate}
  \end{exercise}
  \vspace{.5in}





  \begin{exercise}{9} For each of the following expressions determine the points at which the function is not analytic.\\
    \begin{enumerate}
      \item[b] $\dfrac{iz^3 + 2z}{z^2 + 1}$\\
      \solution This expression is an example of a complex rational expression. We know that it is undefined, and therefore 
      not differentiable when the denominator is zero. Solving for when the denominator is zero, 
      \begin{align*}
        z^2 + 1 &= 0,\\
        z^2 &= -1,\\ 
        z  &=  \sqrt{-1} = i. 
      \end{align*}
      \vspace{.15in}


      \item[d] $z^2(2z^2 - 3z + 1)^{-1}$\\
      \solution Again this expression is a complex rational expression. Solving for when the denominator is equal to zero using the quadratic formula we get the 
      following, 
      \begin{align*}
        z &= \dfrac{3 \pm \sqrt{(-3)^2 - 4(2)(1)}}{2(2)},\\
         &= \dfrac{3 \pm \sqrt{9 - 8}}{4},\\
         &= \dfrac{3 \pm 1 }{4}.
      \end{align*}
    \end{enumerate}
  \end{exercise}
  \vspace{.5in}



  \begin{exercise}{11} Discuss the analyticity of each of the following functions, 
    \begin{enumerate}
      \item[d] $x^2 - y^2 + 2xyi$. \\
      \solution This function is a polynomial and is therefore differentiable on $\mathbb{C}$. 
      Thus the function is analytic and entire.  
      \vspace{.15in} 



      \item[f] $(x + \frac{x}{x^2 + y^2}) + i(y - \frac{y}{x^2 + y^2})$\\
      \solution First consider simplifying the function, 
      \begin{align*}
        (x + \frac{x}{x^2 + y^2}) + i(y - \frac{y}{x^2 + y^2}) &= x + iy + \frac{x}{x^2 + y^2}- \frac{iy}{x^2 + y^2}\\
        &= z + \frac{\bar{z}}{|z|^2}.
      \end{align*}
      Therefore the function is undefined and not analytic when $|z|^2 = 0$. Thus the function is analytic everywhere except 
      $z = 0$.
    \end{enumerate}
  \end{exercise}
  \vspace{.15in}

  \section*{2.4}

  \begin{exercise}{1} Use the Cauchy-Riemann equations to show that the following function are nowhere differentiable.
    \begin{enumerate}
      \item{b} $w = Re z$\\
      \solution Let $z = x + iy$ and note that $w = Re z = x$. Checking the Cauchy-Riemann equations we get that, 
      \begin{equation*}
        \dfrac{du}{dx} = 1,
      \end{equation*}
      \begin{equation*}
        \dfrac{dv}{dy} = 0.
      \end{equation*}
      Since $\dfrac{du}{dx} \neq \dfrac{dv}{dy}$ we know that function is nowhere differentiable. 
      \vspace{.15in}

      \item{c} $w = 2y + ix$\\
      \solution Checking the Cauchy-Riemann equations we get that, 
      \begin{equation*}
        \dfrac{du}{dx} = 0,
      \end{equation*}
      \begin{equation*}
        \dfrac{dv}{dy} = 0.
      \end{equation*}
      Considering the next pair of partial derivatives, 
      \begin{equation*}
        \dfrac{du}{dy} = 2,
      \end{equation*}
      \begin{equation*}
        -\dfrac{dv}{dx} = -1.
      \end{equation*}
      Since $\dfrac{du}{dy} \neq -\dfrac{dv}{dx}$ we know that the function is nowhere differentiable. 
    \end{enumerate}
  \end{exercise}
  \vspace{.5in}



\begin{exercise}{2} Show that $h(z) = x^3 + 3xy^2 - 3x + i(y^3 + 3x^2y - 3y)$ is differentiable on the coordinate axes but 
  is nowhere analytic. \\
  \solution Consider the Cauchy-Riemann equations,  
  \begin{equation*}
    \dfrac{du}{dx} = 3x^2 + 3y^2 - 3,
  \end{equation*}
  \begin{equation*}
    \dfrac{dv}{dy} = 3x^2 + 3y^2 - 3.
  \end{equation*}
  Considering the next pair of partial derivatives, 
  \begin{equation*}
    \dfrac{du}{dy} = 6xy,
  \end{equation*}
  \begin{equation*}
    -\dfrac{dv}{dx} = -6xy.
  \end{equation*}
  Note that $\dfrac{du}{dy} \neq -\dfrac{dv}{dx}$ accept in the case when either $x = 0$ or $y = 0$.  
  The function satisfies the Cauchy-Riemann equations are satisfied when $x = 0$ or $y = 0$, thus the function is differentiable on the coordinate axes.

\end{exercise}
\vspace{.5in}




\begin{exercise}{3}Use Theorem 5 to show that $g(z) = 3x^2 + 2x - 3y^2 - 1 + i(6xy + 2y)$ is entire. Write the 
  function in terms of $z$.\\
  \solution Consider the Cauchy-Riemann equation,
  \begin{equation*}
    \dfrac{du}{dx} =  6x + 2,
  \end{equation*}
  \begin{equation*}
    \dfrac{dv}{dy} = 6x + 2.
  \end{equation*}
  Considering the next pair of partial derivatives, 
  \begin{equation*}
    \dfrac{du}{dy} = -6y,
  \end{equation*}
  \begin{equation*}
    -\dfrac{dv}{dx} = -6y.
  \end{equation*}
  Note that the first partial derivatives are continuous and satisfy the Cauchy-Riemann equations at every points in the plane. 
  Hence by Theorem 5 $g(z)$ is entire. Simplifying the function to get it in terms of $z$, 
  \begin{align*}
    g(z) &= 3x^2 + 2x - 3y^2 - 1 + i(6xy + 2y),\\
    &= 3x^2 + 2x - 3y^2 - 1 + i6xy + i2y,\\
    &= 3x^2 + i6xy - 3y^2 + 2x + i2y - 1,\\
    &= 3(x^2 + i2xy - y^2) + 2(x + iy) - 1,\\
    &= 3(x + iy)^2 + 2(x + iy) - 1, \\
    &= 3(z)^2 + 2(z) - 1. 
  \end{align*}
\end{exercise}
\vspace{.5in}


\begin{exercise}{5} Show that $f(z) = e^{x^2 - y^2}[cos(2xy) + i sin(2xy)]$ is entire and find its derivative.\\
  \solution Consider the Cauchy-Riemann equation,
  \begin{equation*}
    \dfrac{du}{dx} = e^{x^2 - y^2}\sin(2xy)(-2y) + \cos(2xy)e^{x^2 - y^2}(2x),
  \end{equation*}
  \begin{equation*}
    \dfrac{dv}{dy} = e^{x^2 - y^2}\cos(2xy)(2x) + \sin(2xy)e^{x^2 - y^2}(-2y).
  \end{equation*}
  Considering the next pair of partial derivatives, 
  \begin{equation*}
    \dfrac{du}{dy} = e^{x^2 - y^2}\sin(2xy)(-2x) + \cos(2xy)e^{x^2 - y^2}(-2y),
  \end{equation*}
  \begin{equation*}
    -\dfrac{dv}{dx} =  -(e^{x^2 - y^2}\cos(2xy)(2y) + \sin(2xy)e^{x^2 - y^2}(2x)).
  \end{equation*}
  Note that the first partial derivatives are continuous and satisfy the Cauchy-Riemann equations at every points in the plane. 
  Hence by Theorem 5 $g(z)$ is entire. Computing the derivative we get, 
  \begin{align*}
    f'(z) &= \dfrac{du}{dx} + \dfrac{dv}{dx}i\\
     &= e^{x^2 - y^2}\sin(2xy)(-2y) + \cos(2xy)e^{x^2 - y^2}(2x) + e^{x^2 - y^2}\cos(2xy)(2y) + \sin(2xy)e^{x^2 - y^2}(2x)i
  \end{align*}
\end{exercise}
\vspace{.5in}



















\end{document}


















