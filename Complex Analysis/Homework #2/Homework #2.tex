
%%%%%%%%%%%%%%%%%%%%%%%%%%%%%%%%%%%%%%%%%%%%%%%%%%%%%%%%%%%%%%%%%%%%%%%%%%%%%%%%%%%%%%%
%%%%%%%%%%%%%%%%%%%%%%%%%%%%%%%%%%%%%%%%%%%%%%%%%%%%%%%%%%%%%%%%%%%%%%%%%%%%%%%%%%%%%%%
% 
% This top part of the document is called the 'preamble'.  Modify it with caution!
%
% The real document starts below where it says 'The main document starts here'.

\documentclass[12pt]{article}

\usepackage{amssymb,amsmath,amsthm}
\usepackage[top=1in, bottom=1in, left=1.25in, right=1.25in]{geometry}
\usepackage{fancyhdr}
\usepackage{enumerate}
\usepackage{listings}
\usepackage{graphicx}
\usepackage{float}
% Comment the following line to use TeX's default font of Computer Modern.
\usepackage{times,txfonts}



\makeatletter
\renewcommand*\env@matrix[1][*\c@MaxMatrixCols c]{%
  \hskip -\arraycolsep
  \let\@ifnextchar\new@ifnextchar
  \array{#1}}
\makeatother

\newtheoremstyle{homework}% name of the style to be used
  {18pt}% measure of space to leave above the theorem. E.g.: 3pt
  {12pt}% measure of space to leave below the theorem. E.g.: 3pt
  {}% name of font to use in the body of the theorem
  {}% measure of space to indent
  {\bfseries}% name of head font
  {:}% punctuation between head and body
  {2ex}% space after theorem head; " " = normal interword space
  {}% Manually specify head
\theoremstyle{homework} 

% Set up an Exercise environment and a Solution label.
\newtheorem*{exercisecore}{Exercise \@currentlabel}
\newenvironment{exercise}[1]
{\def\@currentlabel{#1}\exercisecore}
{\endexercisecore}

\newcommand{\localhead}[1]{\par\smallskip\noindent\textbf{#1}\nobreak\\}%
\newcommand\solution{\localhead{Solution:}}

%%%%%%%%%%%%%%%%%%%%%%%%%%%%%%%%%%%%%%%%%%%%%%%%%%%%%%%%%%%%%%%%%%%%%%%%
%
% Stuff for getting the name/document date/title across the header
\makeatletter
\RequirePackage{fancyhdr}
\pagestyle{fancy}
\fancyfoot[C]{\ifnum \value{page} > 1\relax\thepage\fi}
\fancyhead[L]{\ifx\@doclabel\@empty\else\@doclabel\fi}
\fancyhead[C]{\ifx\@docdate\@empty\else\@docdate\fi}
\fancyhead[R]{\ifx\@docauthor\@empty\else\@docauthor\fi}
\headheight 15pt

\def\doclabel#1{\gdef\@doclabel{#1}}
\doclabel{Use {\tt\textbackslash doclabel\{MY LABEL\}}.}
\def\docdate#1{\gdef\@docdate{#1}}
\docdate{Use {\tt\textbackslash docdate\{MY DATE\}}.}
\def\docauthor#1{\gdef\@docauthor{#1}}
\docauthor{Use {\tt\textbackslash docauthor\{MY NAME\}}.}
\makeatother

% Shortcuts for blackboard bold number sets (reals, integers, etc.)
\newcommand{\Reals}{\ensuremath{\mathbb R}}
\newcommand{\Nats}{\ensuremath{\mathbb N}}
\newcommand{\Ints}{\ensuremath{\mathbb Z}}
\newcommand{\Rats}{\ensuremath{\mathbb Q}}
\newcommand{\Cplx}{\ensuremath{\mathbb C}}
%% Some equivalents that some people may prefer.
\let\RR\Reals
\let\NN\Nats
\let\II\Ints
\let\CC\Cplx

%%%%%%%%%%%%%%%%%%%%%%%%%%%%%%%%%%%%%%%%%%%%%%%%%%%%%%%%%%%%%%%%%%%%%%%%%%%%%%%%%%%%%%%
%%%%%%%%%%%%%%%%%%%%%%%%%%%%%%%%%%%%%%%%%%%%%%%%%%%%%%%%%%%%%%%%%%%%%%%%%%%%%%%%%%%%%%%
% 
% The main document start here.

% The following commands set up the material that appears in the header.
\doclabel{Math 410: Homework 2}
\docauthor{Stefano Fochesatto}
\docdate{\today}

\begin{document}



\section*{Exercises 1.3}

\begin{exercise}{2} Show that $|z_1z_2z_3| = |z_1||z_2||z_3|$.\\
    \solution Consider the polar form for each complex number $z_1 = r_1( \cos(\theta_1) + i \sin(\theta_1)), z_2 = r_2( \cos(\theta_2) + i \sin(\theta_2))$ and $z_3 = r_3( \cos(\theta_3) + i \sin(\theta_3))$.
    Now consider the product of the three terms and applying equation (7) from the text, we get the following, 
    \begin{equation*}
        z_1z_2z_3 = r_1r_2r_3\left(\cos(\theta_1 + \theta_ 2 +\theta_3) + i \sin(\theta_1 + \theta_ 2 +\theta_3)\right).
    \end{equation*} 
    Thus it follows that $|z_1z_2z_3| = r_1r_2r_3 =  |z_1||z_2||z_3|$. 
\end{exercise}
\vspace{.15in}


\begin{exercise}{4} Show that for any integer $k$. $|z^k| = |z|^k$ (provided that $z \neq 0$ when $k$ is negative).\\
    \solution Suppose some complex number $z$ and let $k \in \mathbb{Z}^+$. Note that the case where $k = 0$ is trivial. Let $k = 1$ and note that, 
    \begin{equation*}
        |z^1| = |z| = |z|^1.
    \end{equation*}
    Suppose that $|z^k| = |z|^k$ for some $k$ and we will proceed by induction on $k$. Consider the term $|z^{k+1}|$ applyting the induction hypothisis we get the following, 
    \begin{align*}
        |z^{k + 1}| &= |z^{k}z|,\\
        &= |z^{k}||z|,\\ 
        &= |z|^{k}|z|,\\
        &= |z|^{k+1}.
    \end{align*}
    For the  $k \in \mathbb{Z}^-$ consider the previous result and with some algebra we get the following, 
    \begin{align*}
        |z^{-k}| &= |\frac{1}{z^k}|,\\
         &= \frac{1}{|z^k|},\\
         &= \frac{1}{|z|^k},\\
         &= |z|^{-k}.
    \end{align*}
\end{exercise}
\vspace{.15in}

\begin{exercise}{5} Find the following, \\
    \begin{enumerate}
        \item[b.] 
        \begin{equation*}
            |\overline{(1 + i)}(2 - 3i)(-3 + 4i)|
        \end{equation*} 
        \solution The result from problem (2) shows us that we can simplify with the following, 
        \begin{equation*}
            |\overline{(1 + i)}(2 - 3i)(-3 + 4i)| = |\overline{(1 + i)}||(2 - 3i)||(-3 + 4i)| = |(1 - i)||(2 - 3i)||(-3 + 4i)|.
        \end{equation*}
        Computing the moduli for each term we get the following, 
        \begin{equation*}
            |\overline{(1 + i)}(2 - 3i)(-3 + 4i)| = \sqrt{2}\sqrt{13}5 = 5\sqrt{26}.
        \end{equation*} 
        \vspace{.15in}
        \item[c.]
        \begin{equation*}
            \left|\dfrac{i(2 + i)^3}{(1 - i)^2}\right|
        \end{equation*}
        \solution Recall equation (12) from the text ((2) from the lecture notes.) which allows us to distribute the moduli through the division. Furthermore our result from problem allows us 
        to extract the exponents from the moduli therefore the problem is simplified to, 
        \begin{equation*}
            \left|\dfrac{i(2 + i)^3}{(1 - i)^2}\right| = \dfrac{|i(2 + i)^3|}{|(1 - i)^2|} = \dfrac{|i||(2 + i)|^3}{|(1 - i)|^2}. 
        \end{equation*}
        Computing the moduli for each term we get, 
        \begin{equation*}
            \left|\dfrac{i(2 + i)^3}{(1 - i)^2}\right| = \dfrac{(1)(\sqrt{5})^3}{(\sqrt{2})^2)} = \dfrac{5\sqrt{5}}{2}.
        \end{equation*}
        \item[d.] 
        \begin{equation*}
            \left|\dfrac{(\pi + i)^{100}}{(\pi - i)^{100}}\right|
        \end{equation*}  
        \solution Applying a similar technique to the previous problem we get the following simplified form, 
        \begin{equation*}
            \left|\dfrac{(\pi + i)^{100}}{(\pi - i)^{100}}\right| = \dfrac{|(\pi + i)|^{100}}{|(\pi - i)|^{100}}. 
        \end{equation*}
        From here it's clear to see that the solution is one since for any complex number $z$ we know that $|z| = |\overline{z}|$. 
    \end{enumerate}
\end{exercise}
\vspace{.15in}




\begin{exercise}{7} Find the argument fo each of the following complex numbers and write each in polar form. \\
    \begin{enumerate}
        \item[b.] $z = -3 + 3i$
        \solution Computing the moduli we get $r = |z| = \sqrt{-3^2 + 3^2} = 3\sqrt{2}$. Computing the argument with the 
        inverse $\tan$ formula we get the following, $\theta = tan^{-1}(\frac{3}{-3}) = \frac{3\pi}{4}$. Therefore we get the following 
        polar form $z = 3\sqrt{2}(\cos(\frac{3\pi}{4}) + i\sin(\frac{3\pi}{4}))$. 
        \item[e.] $z = (1 - i)(-\sqrt{3} + i)$
        \solution Recall from equation (7) in the text that the polar form of a product of two complex numbers can be extracted by multiplying their 
        moduli and summing their arguments. Computing the moduli by multiplying the moduli of each term in the product, $r = r_1r_2 = \sqrt(2)2$. Applying the inverse $\tan$ formula 
        to extract the argument for each term in the product we get the following, 
        \begin{equation*}
            \theta_1 = \tan^{-1}(\frac{-1}{1}) = \frac{3\pi}{4}.
        \end{equation*}
        \begin{equation*}
            \theta_2 = \tan^{-1}(\frac{1}{-\sqrt{3}}) = \tan^{-1}(\frac{1/2}{-\sqrt{3}/2}) = -\frac{\pi}{6}. 
        \end{equation*}
        Thus we get the argument for $z$ through addition, 
        \begin{equation*}
            \theta = \theta_1 + \theta_2 =  \frac{3\pi}{4}-\frac{\pi}{6} = \frac{7\pi}{12}.
        \end{equation*}
        Therefore our final polar form is, 
        \begin{equation*}
            z = 2\sqrt(2)(\cos( \frac{7\pi}{12}) + i\sin(\frac{7\pi}{12})).
        \end{equation*}
        \item[h.] $\dfrac{-\sqrt{7}(1 + i)}{\sqrt{3} + i}$. 
        \solution Recall from equations (10), (11), and (12) we can solve this problem in a similar way to the last one. We begin by computing the 
        moduli for each term in quotient. $r_1 = |-\sqrt{7}||1 + i| = \sqrt{7}\sqrt{2}$ and $r_2 = |\sqrt{3} + i| = 2$. By (12) we know that moduli for $z$ is the quotient of the 
        moduli, thus $r = |z| = \frac{\sqrt{14}}{2}$. Now we compute the argument for each term in the quotient with the inverse $\tan$ formula,
        \begin{equation*}
            \theta_1 = \tan^{-1}(\frac{1}{1}) = \frac{\pi}{4},
        \end{equation*}
        \begin{equation*}
            \theta_2 = \tan^{-1}(\frac{1}{\sqrt{3}}) = \frac{\pi}{6},
        \end{equation*}
        By (11) we get the argument for $z$ by subtraction, 
        \begin{equation*}
            \theta = \theta_1 - \theta_2 = \frac{\pi}{4} - \frac{\pi}{6} = \frac{\pi}{12}.
        \end{equation*}
        Therefore our final polar form is, 
        \begin{equation*}
            z = \frac{\sqrt{14}}{2}(\cos(\frac{\pi}{12}) + i\sin(\frac{\pi}{12}).
        \end{equation*}
    \end{enumerate}
    
\end{exercise}
\vspace{.15in}




\begin{exercise}{12} Find the following, 
    \begin{enumerate}
        \item[a.] $Arg(-6-6i)$
        \solution Applying the inverse $\tan()$ formula we get the following, 
        \begin{equation*}
            Arg(-6-6i) = \tan^{-1}(\frac{-6}{-6}) = \frac{5\pi}{4}.
        \end{equation*}   
        \item[c.] $Arg(10i)$
        \solution Since our complex number only point in the positive imaginary direction we know that it's argument is $\theta = \frac{\pi}{2}$.  
    \end{enumerate}
\end{exercise}
\vspace{.15in}


\begin{exercise}{13} Decide which of the following statements are always true.\\
    \begin{enumerate}
        \item[a.] $Arg(z_1z_2) = Arg(z_1) + Arg(z_2)$ if $z_1, z_2 \neq 0$\\
        \solution False. Let $z_1 = z_2 = -1 + 0i$ Note that $Arg(z_1z_2) = 0 \neq Arg(z_1) + Arg(z_2) = 2\pi$.
        \item[b.] $Arg(\overline{z}) = -Arg(z)$ if $z$ is not a real number.
        \solution True. This is shown by equation (14) in the text.  
        \item[c.] $Arg(z_1/z_2) = Arg(z_1) - Arg(z_2)$ if $z_1, z_2 \neq 0$\\
        \solution False. Let $z_1 = 0 + i$ and $z_2 = 0 - i$. Note that $z_1/z_2 = -1$ so clearly $Arg(z_1/z_2) = \pi$, however we get, 
        $ Arg(z_1) - Arg(z_2) = \frac{\pi}{2} - \frac{3\pi}{2} = -\pi$.
        \item[d.] $arg(z) = Arg(z + 2\pi k)$ where $k \in \mathbb{Z}$ if $z \neq 0$.\\
        \solution True. This definition is stated explicitly in the footnotes of section 1.3 after equation (5).  
    \end{enumerate}
\end{exercise}


\newpage
\section*{Exercises 1.4}

\begin{exercise}{2} Write the given numbers in standard form.\\
    \begin{enumerate}
        \item[b.] $z = 2e^{3 + i\pi/6}$.
        \solution Simplifying $z$ to pull the argument and moduli
        \begin{equation*}
            z = 2e^{3 + i\pi/6} = (2e^3)e^{i\pi/6}.
        \end{equation*}
        Since $\pi/6$ is the argument and the moduli is $2e^3$ we know that the standard form of $z$ looks like, 
        $z = 2e^3(\frac{\sqrt{3}}{2} + \frac{1}{2}i) = 3e^3+ e^3i$.



        \item[c.] $z = e^x$ where $x = 4e^{i\pi/3}$.
        \solution First note that we can simplify the inner exponential(x) with the fact that the moduli is $4$ and the argument is 
        $\pi/3$ we get the following, $x = 4(\frac{1}{2} + \frac{\sqrt{3}}{2}i) = 2 + 2\sqrt{3}i$. No we simplify the outer exponential,
        \begin{equation*}
            z = e^{2 + 2\sqrt{3}i} =e^2e^{2\sqrt{3}i}.
        \end{equation*}
        Applying the applying equation (7) to extract the components of the standard form from an argument of $2\sqrt{3}$ we get, 
        \begin{equation*}
            z = e^2(\cos(2\sqrt{3}) + i \sin(2\sqrt{3}))
        \end{equation*}
    \end{enumerate}
    
\end{exercise}
\vspace{.15in}





\begin{exercise}{4} Write each of the given numbers in polar form $re^{i\theta}$.\\
    \begin{enumerate}
        \item[b.] $z = \dfrac{2 + 2i}{-\sqrt{3} + i}$.
        \solution  Simplifying the numerator and denominator by factoring out the moduli then using the unit circle to pull the argument we get, 
        \begin{equation*}
            z = \dfrac{2 + 2i}{-\sqrt{3} + i} = \dfrac{2\sqrt{2}(\frac{1}{\sqrt{2}} + \frac{1}{\sqrt{2}})i}{2(-\frac{\sqrt{3}}{2} + \frac{1}{2})i} = \dfrac{2\sqrt{2}e^{i\pi/4}}{2e^{i5\pi/6}} = \sqrt{2}e^{(i(\pi/4 - 5\pi/6))} = \sqrt{2}e^{(i(-7\pi/12))}
        \end{equation*}
        
        
        \item[c.] $z = \dfrac{2i}{3e^{4 + i}}$.\\
        \solution Continuing in a similar fashion we get, 
        \begin{equation*}
            z = \dfrac{2i}{3e^{4 + i}} =\dfrac{2e^{i(\pi/2)}}{3e^{4}e^i} = \dfrac{2}{3e^4} e^{i(2\pi-1)}.  
        \end{equation*}
    \end{enumerate}
    
\end{exercise}
\vspace{.15in}




\begin{exercise}{5} Show that $|e^{x + iy}| = e^x$ and arg $e^{x + iy} = y + 2k\pi$ for $k \in \mathbb{Z}$.\\
    \solution First consider that, 
    \begin{equation*}
        |e^{x + iy}| = |e^{x}e^{iy}| = |e^{x}||e^{iy}| 
    \end{equation*}
    Applying equation (7) from the text we get. 
    \begin{equation*}
        |e^{x + iy}| = |e^{x}||\cos(y) + i\sin(y)| = |e^{x}|(1). 
    \end{equation*} 
    Showing the that the set arg  $e^{x + iy} = y + 2k\pi$ for $k \in \mathbb{Z}$ first consider that the following sets are equivalent, 
    \begin{equation*}
        arg (e^{x + iy}) = arg (e^{iy}) 
    \end{equation*}
    From equation (7) we get the following, 
    \begin{equation*}
        arg (e^{x + iy}) = arg (cos(y) + i sin(y)) 
    \end{equation*}
    Computing the $arg (cos(y) + i sin(y))$ using the inverse tangent formula we get $arg (e^{x + iy}) = \tan^{-1}(\tan(y)) =  y + 2k\pi$ for $k \in \mathbb{Z}$. 
    
\end{exercise}
\vspace{.15in}


\begin{exercise}{7} Show that $e^z = e^{z + 2\pi i}$ for all $z$.\\
    \solution By applying definition (5) to the right hand side we get the following, 
    \begin{equation*}
        e^{z + 2\pi i} = e^{z}(\cos(2\pi) + i \sin(2\pi)) = e^z.
    \end{equation*}
\end{exercise}
\vspace{.15in}


\begin{exercise}{8b} Show that $\overline{e^z} = e^{\overline{z}}$ for all $z$.\\
    \solution Let $z = a + ib$ for some $a, b \in \RR$. Applying definition (5) we get the following, 
    \begin{equation*}
        \overline{e^z} = \overline{e^a\cos(b) + ie^a\sin(b)} =e^a\cos(b) - ie^a\sin(b). 
    \end{equation*}
    Recall the negative angle identities, $\sin(-\theta) = -\sin(\theta)$ and $\cos(-\theta) = \cos(\theta)$ and note that, 
    \begin{equation*}
        e^{\overline{z}} = e^a\cos(-b) + ie^a\sin(-b) =  e^a\cos(b) - ie^a\sin(b).
    \end{equation*}
    Thus it follows that $\overline{e^z} = e^{\overline{z}}$ for all $z$.
\end{exercise}
\vspace{.15in}



\begin{exercise}{9} Show that $(e^z)^n = e^{nz}$ for any integer $n$.\\
    \solution Suppose some complex number $z$ and let $n \in \mathbb{Z}$. Note that the case where $n = 0$ is trivial. Let $n = 1$ and 
    note that, 
    \begin{equation*}
        (e^z)^{(1)} = e^{z} = e^{(1)z}. 
    \end{equation*} 
    Suppose that $(e^z)^n = e^{nz}$ for some $n \in \mathbb{Z}^+$. We will proceed by induction on $n$. Consider $(e^z)^{n+1}$ and note that, 
    \begin{align*}
        (e^z)^{n+1} &= (e^z)^{n} (e^z),\\
         &= e^{nz} (e^z),\\ 
         &= e^{(n+1)z}. 
    \end{align*} 
    For the  $n \in \mathbb{Z}^-$ consider the previous result and with some algebra we get the following, 
    \begin{align*}
        (e^z)^{(-1)}n &= \dfrac{1}{(e^z)^n},\\
        &= \dfrac{1}{e^{nz}},\\
        &= e^{(-1)nz}.
    \end{align*}
\end{exercise}
\vspace{.15in}


\begin{exercise}{10} Show that $|e^z| \leq 1$ if $Re z \leq 0$.\\
    \solution Suppose $z = a + bi$ is a complex number with $a, b \in \RR$ such that $a \leq 0$. Consider the following, 
    \begin{equation*}
        |e^z| = |e^{a + bi}| = |e^a(\cos(b) + i \sin(b))| = e^a.  
    \end{equation*}
    By the definition of the exponential when $Re z = a \leq 0$ we know that $|e^z| = e^a \leq 1$.    
\end{exercise}
\vspace{.15in}



\begin{exercise}{11} Determine which of the following properties of the real exponential function remain 
    true for the complex exponential function (that is, for $x$ replaced by $z$).\\
    \begin{enumerate}
        \item[a.] $e^x$ is never zero.\\
        \solution  Let $z = x + yi$, by the equation (8) we know that $e^z = e^x(\cos(y) + i\sin(y))$. Clearly 
        $e^x > 0$, and we know that $(\cos(y) + i\sin(y)) \neq 0$ Since there is no solution to the system $\cos(y) = 0$, $\sin(y) = 0$. 
        Thus this is true for the complex exponential function. 
        
        \item[b.] $e^x$ is a one-to-one function.\\
        \solution  Consider equation (10) in the text and note that $e^{2\pi i} = e^{-2\pi i}$ thus this is false for the complex exponential function.
        
        \item[c.] $e^x$ is defined for all $x$.\\
        \solution  Recall again the decomposition of $e^z$ via equation (8), $e^z = e^x(\cos(y) + i\sin(y))$. Clearly $e^x$ is well defined 
        and trig functions are also well defined for all $x,y \in \RR$. The product of well defined functions is also well defined thus $e^z$ is well defined. 

        \item[d.] $e^{-x} = 1/e^x$.\\   
        \solution  Again recall the decomposition of $e^z$ via equation (8), note that $e^{-z} = e^{(-1)x}(\cos((-1)y) + i\sin((-1)y))$. by De Moivre's formula (16) we get 
        the following, 
        \begin{equation*}
            e^{-z} = e^{(-1)x}(\cos((-1)y) + i\sin((-1)y)) = \left(e^{x}(\cos(y) + i\sin(y))\right)^{-1} = \dfrac{1}{e^z}. 
        \end{equation*}
    \end{enumerate}
\end{exercise}
\vspace{.15in}


\begin{exercise}{12a} Use De Moivre's formula together with the binomial formula to derive $\sin(3\theta) = 3\cos^2(\theta)\sin(\theta) - \sin^3(\theta)$.\\
    \solution Note that the following is true by De Moivre's formula. 
    \begin{equation*}
        \cos(3\theta) + i\sin(3\theta) = (\cos(\theta) + i\sin(\theta))^3
    \end{equation*} 
    Now applying the binomial formula to compute the expansion we get, 
    \begin{align*}
        \cos(3\theta) + i\sin(3\theta) &= (\cos(\theta) + i\sin(\theta))^3,\\
         &= \cos^3(\theta) + 3\cos^2(\theta)\sin(\theta)i - 3\cos(\theta)\sin^2(\theta) - \sin^3(\theta)i,\\
         &= (\cos^3(\theta) - 3\cos(\theta)\sin^2(\theta)) + (3\cos^2(\theta)\sin(\theta) - \sin^3(\theta))i.\\ 
    \end{align*} 
    Consider the imaginary component of both sides,
    \begin{equation*}
        \sin(3\theta) = 3\cos^2(\theta)\sin(\theta) - \sin^3(\theta)
    \end{equation*}
\end{exercise}
\vspace{.15in}



\begin{exercise}{13} Show how the following trigonometric identities follow from equations (11) and (12).\\
    \begin{enumerate}
        \item[a.] $\sin^2(\theta) + \cos^2(\theta) = 1$\\
        \solution Applying equations (11) and (12) to the right hand side we get the following,
        \begin{align*}
            \sin^2(\theta) + \cos^2(\theta) &= \left(\dfrac{e^{i\theta} - e^{-i\theta}}{2i}\right)^2 + \left(\dfrac{e^{i\theta} + e^{-i\theta}}{2}\right)^2\\
             &= \dfrac{(e^{i\theta} + e^{-i\theta})^2 - (e^{i\theta} - e^{-i\theta})^2}{4}\\
             &= \dfrac{(e^{2i\theta}+2+e^{-2i\theta}) - (e^{2i\theta}-2+e^{-2i\theta})}{4}\\
             &= \dfrac{4}{4} = 1.
        \end{align*} 
        \item[b.] $\cos(\theta_1 + \theta_2) = \cos(\theta_1)\cos(\theta_2) - \sin(\theta_1)\sin(\theta_2)$\\ 
        \solution Applying equation (11) to the left hand side we get, 
        \begin{align*}
            \cos(\theta_1 + \theta_2) &= \dfrac{e^{i(\theta_1 + \theta_2)} + e^{-i(\theta_1 + \theta_2)}}{2}\\
            &=\dfrac{e^{i\theta_1 + i\theta_2} + e^{-i\theta_1 - i\theta_2}}{2}\\
            &=\dfrac{e^{i\theta_1}e^{i\theta_2} + e^{-i\theta_1} e^{-i\theta_2}}{2}
        \end{align*}
        Applying equation (7) to the denominator we get the following, 
        \begin{align*}
            \cos(\theta_1 + \theta_2) &= \dfrac{e^{i\theta_1}e^{i\theta_2} + e^{-i\theta_1} e^{-i\theta_2}}{2},\\
            &=\dfrac{((\cos(\theta_1) + i\sin(\theta_1))(\cos(\theta_2) + i\sin(\theta_2)))}{2},\\ 
            &+ \dfrac{((\cos(-\theta_1) + i\sin(-\theta_1))(\cos(-\theta_2) + i\sin(-\theta_2)))}{2},\\
            &= \dfrac{\cos(\theta_1)\cos(\theta_2) - \sin(\theta_1)\sin(\theta_2) + i\sin(\theta_1)\cos(\theta_2) + i\cos(\theta_1)\sin(\theta_2)}{2},\\
            &+ \dfrac{\cos(\theta_1)\cos(\theta_2) - \sin(\theta_1)\sin(\theta_2) - i\sin(\theta_1)\cos(\theta_2) - i\cos(\theta_1)\sin(\theta_2)}{2},\\
            &=  \cos(\theta_1)\cos(\theta_2) - \sin(\theta_1)\sin(\theta_2).
        \end{align*}
    \end{enumerate}
\end{exercise}
\vspace{.15in}


\begin{exercise}{23b} Compute the following integral by using the representation (11) or (12) together with the 
    binomial formula.
    \begin{equation*}
        \int_0^{2\pi} sin^6(2\theta)d\theta
    \end{equation*}
    \solution Applying the complex representation, 
    \begin{equation*}
        \int_0^{2\pi}\left(\dfrac{e^{i2\theta} - e^{-i2\theta}}{2i}\right)^6 d\theta
    \end{equation*}
    Applying the binomial theorem we can simplify, 
    \begin{align*}
        \int_0^{2\pi}\left(\dfrac{e^{i2\theta} - e^{-i2\theta}}{2i}\right)^6 d\theta &= \dfrac{1}{(2i)^6}\int_0^{2\pi}\left(e^{i2\theta} - e^{-i2\theta}\right)^6 d\theta\\
         &= \dfrac{1}{(2i)^6}\int_0^{2\pi}\left(e^{i2\theta} - e^{-i2\theta}\right)^6 d\theta\\
         &= \dfrac{1}{(2i)^6} \int_0^{2\pi} e^{12i\theta}-6e^{8i\theta}+15e^{4i\theta}-20+15e^{-4i\theta}-6e^{-8i\theta}+e^{-12i\theta} d\theta\\
         &= \dfrac{1}{(2i)^6} \int_0^{2\pi} e^{12i\theta}-6e^{8i\theta}+15e^{4i\theta}-20+15e^{-4i\theta}-6e^{-8i\theta}+e^{-12i\theta} d\theta\\
         &= \dfrac{1}{(2i)^6} \int_0^{2\pi} 2(1)\cos(12\theta)-2(6)\cos(8\theta)+2(15)\cos(4\theta) - 20\\ 
         &+ (i\sin(12\theta)-i\sin(12\theta) + 6i\sin(8\theta) - 6i\sin(8\theta) + 15i\sin(4\theta) - 15i\sin(4\theta))d\theta\\
         &= -\dfrac{1}{64} \int_0^{2\pi} 2\cos(12\theta)-12\cos(8\theta)+30\cos(4\theta) - 20 d\theta     
        \end{align*}
    Integrating each term we get, 
    \begin{align*}
        \int_0^{2\pi} sin^6(2\theta)d\theta &= -\dfrac{1}{64} \int_0^{2\pi} 2\cos(12\theta)-12\cos(8\theta)+30\cos(4\theta) - 20 d\theta,\\
        &= -\dfrac{1}{64} \left[2\left(\dfrac{\sin(12\theta)}{12}\right) - 12\left(\dfrac{\sin(8\theta)}{8}\right) + 30\left(\dfrac{\sin(4\theta)}{4}\right) - 20\theta\right]_0^{2\pi},\\
        &= -\dfrac{1}{64}(-20)(2\pi),\\
        &= \dfrac{40\pi}{64}.
    \end{align*}
\end{exercise}










\end{document}


















