
%%%%%%%%%%%%%%%%%%%%%%%%%%%%%%%%%%%%%%%%%%%%%%%%%%%%%%%%%%%%%%%%%%%%%%%%%%%%%%%%%%%%%%%
%%%%%%%%%%%%%%%%%%%%%%%%%%%%%%%%%%%%%%%%%%%%%%%%%%%%%%%%%%%%%%%%%%%%%%%%%%%%%%%%%%%%%%%
% 
% This top part of the document is called the 'preamble'.  Modify it with caution!
%
% The real document starts below where it says 'The main document starts here'.

\documentclass[12pt]{article}

\usepackage{amssymb,amsmath,amsthm}
\usepackage[top=1in, bottom=1in, left=1.25in, right=1.25in]{geometry}
\usepackage{fancyhdr}
\usepackage{enumerate}
\usepackage{listings}
\usepackage{graphicx}
\usepackage{float}
\usepackage{multicol}
% Comment the following line to use TeX's default font of Computer Modern.
\usepackage{times,txfonts}
\usepackage{mwe}
\usepackage{caption}
\usepackage{subcaption}





\makeatletter
\renewcommand*\env@matrix[1][*\c@MaxMatrixCols c]{%
  \hskip -\arraycolsep
  \let\@ifnextchar\new@ifnextchar
  \array{#1}}
\makeatother

\newtheoremstyle{homework}% name of the style to be used
  {18pt}% measure of space to leave above the theorem. E.g.: 3pt
  {12pt}% measure of space to leave below the theorem. E.g.: 3pt
  {}% name of font to use in the body of the theorem
  {}% measure of space to indent
  {\bfseries}% name of head font
  {:}% punctuation between head and body
  {2ex}% space after theorem head; " " = normal interword space
  {}% Manually specify head
\theoremstyle{homework} 

% Set up an Exercise environment and a Solution label.
\newtheorem*{exercisecore}{Exercise \@currentlabel}
\newenvironment{exercise}[1]
{\def\@currentlabel{#1}\exercisecore}
{\endexercisecore}

\newcommand{\localhead}[1]{\par\smallskip\noindent\textbf{#1}\nobreak\\}%
\newcommand\solution{\localhead{Solution:}}

%%%%%%%%%%%%%%%%%%%%%%%%%%%%%%%%%%%%%%%%%%%%%%%%%%%%%%%%%%%%%%%%%%%%%%%%
%
% Stuff for getting the name/document date/title across the header
\makeatletter
\RequirePackage{fancyhdr}
\pagestyle{fancy}
\fancyfoot[C]{\ifnum \value{page} > 1\relax\thepage\fi}
\fancyhead[L]{\ifx\@doclabel\@empty\else\@doclabel\fi}
\fancyhead[C]{\ifx\@docdate\@empty\else\@docdate\fi}
\fancyhead[R]{\ifx\@docauthor\@empty\else\@docauthor\fi}
\headheight 15pt

\def\doclabel#1{\gdef\@doclabel{#1}}
\doclabel{Use {\tt\textbackslash doclabel\{MY LABEL\}}.}
\def\docdate#1{\gdef\@docdate{#1}}
\docdate{Use {\tt\textbackslash docdate\{MY DATE\}}.}
\def\docauthor#1{\gdef\@docauthor{#1}}
\docauthor{Use {\tt\textbackslash docauthor\{MY NAME\}}.}
\makeatother

% Shortcuts for blackboard bold number sets (reals, integers, etc.)
\newcommand{\Reals}{\ensuremath{\mathbb R}}
\newcommand{\Nats}{\ensuremath{\mathbb N}}
\newcommand{\Ints}{\ensuremath{\mathbb Z}}
\newcommand{\Rats}{\ensuremath{\mathbb Q}}
\newcommand{\Cplx}{\ensuremath{\mathbb C}}
%% Some equivalents that some people may prefer.
\let\RR\Reals
\let\NN\Nats
\let\II\Ints
\let\CC\Cplx
%%%%%%%%%%%%%%%%%%%%%%%%%%%%%%%%%%%%%%%%%%%%%%%%%%%%%%%%%%%%%%%%%%%%%%%%%%%%%%%%%%%%%%%
%%%%%%%%%%%%%%%%%%%%%%%%%%%%%%%%%%%%%%%%%%%%%%%%%%%%%%%%%%%%%%%%%%%%%%%%%%%%%%%%%%%%%%%
% 
% The main document start here.




%  \textbf{Code:}
%  \begin{center}
%  \lstinputlisting[basicstyle = \footnotesize]{}
%  \end{center}
%  
%  \begin{footnotesize}
%  \begin{verbatim}
%    
%  \end{verbatim}
%  \end{footnotesize}
%  
%  
%  \begin{figure}[H]
%    \begin{center}
%      \caption{}
%    \includegraphics[width = \textwidth]{}
%    \end{center}
%  \end{figure}





% The following commands set up the material that appears in the header.
\doclabel{Stat 605: Exam 2}
\docauthor{Stefano Fochesatto}
\docdate{\today}

\begin{document}

\begin{exercise}{1} Definitions. For each of the following, define the term and stat its importance in statistics (spatial 
  statistics if the term is specific to spatial stats). (I expect 2-3 sentences for each of these, no more.)\\
  \begin{enumerate}
    \item[(a)] edge effects (for point pattern data)\\
    \solution 
    \vspace{.15in} 
    \item[(b)] CSR (complete spatial randomness)\\
    \solution 
    \vspace{.15in} 
    \item[(c)] Monte Carlo tests (also, why are they so useful when working with point pattern data?)\\
    \solution    
  \end{enumerate} 
\end{exercise}
\newpage


\begin{exercise}{2} We model CSR using a spatial Poisson process (for point pattern data). Consider a rectangular region $R$
  with $0 \geq x \geq 3$ and $0 \geq y \geq 2$.\\
  \begin{enumerate}
    \item[(a)] If the intensity for a (homogenous) Poisson process in this region is given by $\lambda(x, y) = 1.4,$\\
    \begin{enumerate}
      \item[i.] What is the distribution of $N(R),$ the number of events in the region?\\
      \solution
      \vspace{.15in}
      \item[ii.] Find $P(N(R)) = 12,$ the probability that there are 12 events in the region.\\
      \solution
      \vspace{.15in}
    \end{enumerate} 

    \item[(b)] If the intensity of the inhomogeneous Poisson process in this region is $\lambda(x, y) = x + y$, \\
    \begin{enumerate}
      \item[i.] Calculate $\gamma = \iint_R \lambda(x, y) dx dy$\\
      \solution
      \vspace{.15in}
      \item[ii.] Find the distribution of $N(R)$\\
      \solution
      \vspace{.15in}
     \item[iii.] What is the expected number of event sin the region R?\\
     \solution
     \vspace{.15in}
    \end{enumerate} 
  \end{enumerate}
  
\end{exercise}







\end{document}


















