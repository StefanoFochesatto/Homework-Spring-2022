
%%%%%%%%%%%%%%%%%%%%%%%%%%%%%%%%%%%%%%%%%%%%%%%%%%%%%%%%%%%%%%%%%%%%%%%%%%%%%%%%%%%%%%%
%%%%%%%%%%%%%%%%%%%%%%%%%%%%%%%%%%%%%%%%%%%%%%%%%%%%%%%%%%%%%%%%%%%%%%%%%%%%%%%%%%%%%%%
% 
% This top part of the document is called the 'preamble'.  Modify it with caution!
%
% The real document starts below where it says 'The main document starts here'.

\documentclass[12pt]{article}

\usepackage{amssymb,amsmath,amsthm}
\usepackage[top=1in, bottom=1in, left=1.25in, right=1.25in]{geometry}
\usepackage{fancyhdr}
\usepackage{enumerate}
\usepackage{listings}
\usepackage{graphicx}
\usepackage{float}
\usepackage{multicol}
% Comment the following line to use TeX's default font of Computer Modern.
\usepackage{times,txfonts}
\usepackage{mwe}
\usepackage{caption}
\usepackage{subcaption}





\makeatletter
\renewcommand*\env@matrix[1][*\c@MaxMatrixCols c]{%
  \hskip -\arraycolsep
  \let\@ifnextchar\new@ifnextchar
  \array{#1}}
\makeatother

\newtheoremstyle{homework}% name of the style to be used
  {18pt}% measure of space to leave above the theorem. E.g.: 3pt
  {12pt}% measure of space to leave below the theorem. E.g.: 3pt
  {}% name of font to use in the body of the theorem
  {}% measure of space to indent
  {\bfseries}% name of head font
  {:}% punctuation between head and body
  {2ex}% space after theorem head; " " = normal interword space
  {}% Manually specify head
\theoremstyle{homework} 

% Set up an Exercise environment and a Solution label.
\newtheorem*{exercisecore}{Exercise \@currentlabel}
\newenvironment{exercise}[1]
{\def\@currentlabel{#1}\exercisecore}
{\endexercisecore}

\newcommand{\localhead}[1]{\par\smallskip\noindent\textbf{#1}\nobreak\\}%
\newcommand\solution{\localhead{Solution:}}

%%%%%%%%%%%%%%%%%%%%%%%%%%%%%%%%%%%%%%%%%%%%%%%%%%%%%%%%%%%%%%%%%%%%%%%%
%
% Stuff for getting the name/document date/title across the header
\makeatletter
\RequirePackage{fancyhdr}
\pagestyle{fancy}
\fancyfoot[C]{\ifnum \value{page} > 1\relax\thepage\fi}
\fancyhead[L]{\ifx\@doclabel\@empty\else\@doclabel\fi}
\fancyhead[C]{\ifx\@docdate\@empty\else\@docdate\fi}
\fancyhead[R]{\ifx\@docauthor\@empty\else\@docauthor\fi}
\headheight 15pt

\def\doclabel#1{\gdef\@doclabel{#1}}
\doclabel{Use {\tt\textbackslash doclabel\{MY LABEL\}}.}
\def\docdate#1{\gdef\@docdate{#1}}
\docdate{Use {\tt\textbackslash docdate\{MY DATE\}}.}
\def\docauthor#1{\gdef\@docauthor{#1}}
\docauthor{Use {\tt\textbackslash docauthor\{MY NAME\}}.}
\makeatother

% Shortcuts for blackboard bold number sets (reals, integers, etc.)
\newcommand{\Reals}{\ensuremath{\mathbb R}}
\newcommand{\Nats}{\ensuremath{\mathbb N}}
\newcommand{\Ints}{\ensuremath{\mathbb Z}}
\newcommand{\Rats}{\ensuremath{\mathbb Q}}
\newcommand{\Cplx}{\ensuremath{\mathbb C}}
%% Some equivalents that some people may prefer.
\let\RR\Reals
\let\NN\Nats
\let\II\Ints
\let\CC\Cplx
%%%%%%%%%%%%%%%%%%%%%%%%%%%%%%%%%%%%%%%%%%%%%%%%%%%%%%%%%%%%%%%%%%%%%%%%%%%%%%%%%%%%%%%
%%%%%%%%%%%%%%%%%%%%%%%%%%%%%%%%%%%%%%%%%%%%%%%%%%%%%%%%%%%%%%%%%%%%%%%%%%%%%%%%%%%%%%%
% 
% The main document start here.




%  \textbf{Code:}
%  \begin{center}
%  \lstinputlisting[basicstyle = \footnotesize]{}
%  \end{center}
%  
%  \begin{footnotesize}
%  \begin{verbatim}
%    
%  \end{verbatim}
%  \end{footnotesize}
%  
%  
%  \begin{figure}[H]
%    \begin{center}
%      \caption{}
%    \includegraphics[width = \textwidth]{}
%    \end{center}
%  \end{figure}





% The following commands set up the material that appears in the header.
\doclabel{Stat 605: Homework 7}
\docauthor{Stefano Fochesatto}
\docdate{\today}

\begin{document}

\begin{exercise}{1} Conduct the quadrat test for complete spatial randomness on the cells data set. Carry out the analysis using 
  a 2x2 grid, a 3x3 grid, and a 4x4 grid. Is there evidence of regularity or clustering on one scale but not the others? Use $\alpha = .05$, 
  and carry out just 2-sided test.\\
  Please summarize your results, including a table of counts for each for each of the grid sizes and data sets.\\
  the cell data is included in the spatstat package. 

  \begin{footnotesize}
  \begin{verbatim}
    library(spatstat)
    data(cells)
  \end{verbatim}
  \end{footnotesize}  

  Sample code, which you'll need to modify:

  \begin{footnotesize}
  \begin{verbatim}
    tmp <- quadratcount(cells, nx=2, ny=2 )
    tmp
    plot(tmp)
    quadrat.test(cells, nx=2, ny=2, alternative="two.sided")
  \end{verbatim}
  \end{footnotesize}  
  \solution Let's recall a few ideas behind conducting a formal test for CSR using quadrat data. First note that we obtain counts $y_i$
  for each of the $m$ quadrats, and remember that as a rule of thumb it is suggested that we have $m > 6$ and $y_i > 5$. Our test statistic, the index of dispersion (or $\chi^2$ goodness of fit)
  is computed by,  
  \begin{equation*}
    I = \dfrac{(m-1)s^2}{\bar{y}}. 
  \end{equation*}
  Note that $I$ is approximately distributed by a $\chi^2$ with $(m-1)$ degrees of freedom. Computing the p-values for the various tests we get,
  two-tailed (2x the smaller direction), 
  \begin{align*}
    H_0&: \text{CSR}\\
    H_a&: \text{Not CSR}\\
  \end{align*}
  right-tailed, 
  \begin{align*}
    H_0&: \text{CSR}\\
    H_a&: \text{Clustered}\\
  \end{align*}
  and left-tailed, 
  \begin{align*}
    H_0&: \text{CSR}\\
    H_a&: \text{Regular}\\
  \end{align*}
  Conducting the tests in R we get the following counts, p-values, and test statistics, 

  \begin{center}
    \begin{tabular}{c|| c c c}
        & 2 & 3 & 4\\
      \hline 
      p-value        & 0.1518  &  0.3391 & 0.0007283\\
      test statistic & 0.47619 & 4.2857  & 2.9524\\
      count          &         &           & 2\\         
                                           & 3\\         
                                           & 3\\       
                                           & 2\\
                                           & 2\\         
                                           & 3\\         
                                           & 4\\       
                                           & 3\\
                                           & 2\\         
                                           & 2\\         
                                           & 2\\       
                                           & 2\\
                                           & 3\\         
                                           & 4\\         
                                           & 2\\       
                                           & 3





     \end{tabular}
    \end{center}


























\end{exercise}







\end{document}


















