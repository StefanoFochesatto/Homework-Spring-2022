
%%%%%%%%%%%%%%%%%%%%%%%%%%%%%%%%%%%%%%%%%%%%%%%%%%%%%%%%%%%%%%%%%%%%%%%%%%%%%%%%%%%%%%%
%%%%%%%%%%%%%%%%%%%%%%%%%%%%%%%%%%%%%%%%%%%%%%%%%%%%%%%%%%%%%%%%%%%%%%%%%%%%%%%%%%%%%%%
% 
% This top part of the document is called the 'preamble'.  Modify it with caution!
%
% The real document starts below where it says 'The main document starts here'.

\documentclass[12pt]{article}

\usepackage{amssymb,amsmath,amsthm}
\usepackage[top=1in, bottom=1in, left=1.25in, right=1.25in]{geometry}
\usepackage{fancyhdr}
\usepackage{enumerate}
\usepackage{listings}
\usepackage{graphicx}
\usepackage{float}

\usepackage{mwe}
\usepackage{caption}
\usepackage{subcaption}
% Comment the following line to use TeX's default font of Computer Modern.
\usepackage{times,txfonts}



\makeatletter
\renewcommand*\env@matrix[1][*\c@MaxMatrixCols c]{%
  \hskip -\arraycolsep
  \let\@ifnextchar\new@ifnextchar
  \array{#1}}
\makeatother

\newtheoremstyle{homework}% name of the style to be used
  {18pt}% measure of space to leave above the theorem. E.g.: 3pt
  {12pt}% measure of space to leave below the theorem. E.g.: 3pt
  {}% name of font to use in the body of the theorem
  {}% measure of space to indent
  {\bfseries}% name of head font
  {:}% punctuation between head and body
  {2ex}% space after theorem head; " " = normal interword space
  {}% Manually specify head
\theoremstyle{homework} 

% Set up an Exercise environment and a Solution label.
\newtheorem*{exercisecore}{Exercise \@currentlabel}
\newenvironment{exercise}[1]
{\def\@currentlabel{#1}\exercisecore}
{\endexercisecore}

\newcommand{\localhead}[1]{\par\smallskip\noindent\textbf{#1}\nobreak\\}%
\newcommand\solution{\localhead{Solution:}}

%%%%%%%%%%%%%%%%%%%%%%%%%%%%%%%%%%%%%%%%%%%%%%%%%%%%%%%%%%%%%%%%%%%%%%%%
%
% Stuff for getting the name/document date/title across the header
\makeatletter
\RequirePackage{fancyhdr}
\pagestyle{fancy}
\fancyfoot[C]{\ifnum \value{page} > 1\relax\thepage\fi}
\fancyhead[L]{\ifx\@doclabel\@empty\else\@doclabel\fi}
\fancyhead[C]{\ifx\@docdate\@empty\else\@docdate\fi}
\fancyhead[R]{\ifx\@docauthor\@empty\else\@docauthor\fi}
\headheight 15pt

\def\doclabel#1{\gdef\@doclabel{#1}}
\doclabel{Use {\tt\textbackslash doclabel\{MY LABEL\}}.}
\def\docdate#1{\gdef\@docdate{#1}}
\docdate{Use {\tt\textbackslash docdate\{MY DATE\}}.}
\def\docauthor#1{\gdef\@docauthor{#1}}
\docauthor{Use {\tt\textbackslash docauthor\{MY NAME\}}.}
\makeatother

% Shortcuts for blackboard bold number sets (reals, integers, etc.)
\newcommand{\Reals}{\ensuremath{\mathbb R}}
\newcommand{\Nats}{\ensuremath{\mathbb N}}
\newcommand{\Ints}{\ensuremath{\mathbb Z}}
\newcommand{\Rats}{\ensuremath{\mathbb Q}}
\newcommand{\Cplx}{\ensuremath{\mathbb C}}
%% Some equivalents that some people may prefer.
\let\RR\Reals
\let\NN\Nats
\let\II\Ints
\let\CC\Cplx
%%%%%%%%%%%%%%%%%%%%%%%%%%%%%%%%%%%%%%%%%%%%%%%%%%%%%%%%%%%%%%%%%%%%%%%%%%%%%%%%%%%%%%%
%%%%%%%%%%%%%%%%%%%%%%%%%%%%%%%%%%%%%%%%%%%%%%%%%%%%%%%%%%%%%%%%%%%%%%%%%%%%%%%%%%%%%%%
% 
% The main document start here.

% The following commands set up the material that appears in the header.



%  \textbf{Code:}
%  \begin{center}
%  \lstinputlisting[basicstyle = \footnotesize]{}
%  \end{center}
%  
%  \begin{footnotesize}
%  \begin{verbatim}
%    
%  \end{verbatim}
%  \end{footnotesize}
%  
%  
%  \begin{figure}[H]
%    \begin{center}
%      \caption{}
%    \includegraphics[width = \textwidth]{}
%    \end{center}
%  \end{figure}




\doclabel{Stat 461: Homework 10}
\docauthor{Stefano Fochesatto}
\docdate{\today}

\begin{document}
\begin{exercise}{1}
  
  There is a dataset called banknote\_authentication that is in the package nonet, from Dr. Volker Lohweg. It is also available at the UCI machine
  learning repository\\
  The variables are:
  \begin{enumerate}
    \item variance of wavelet transformed image
    \item skewness of wavelet transformed image
    \item kurtosis of wavelet transformed image
    \item entropy of image
    \item class (0 is genuine, 1 is forgery).
  \end{enumerate}

  This research started with a classical approach to image analysis. You take the image, convert it to grey scale, then apply a mathematical
  transform. In this case the transform is a wavelet, other options include Fourier transforms of various types. They encode information about
  similarity of close-together pixels. To give you an idea of how wavelet transforms work (in the simplest possible case), I’ll have you do a 1-D
  wavelet transform of a particularly simple kind - the Haar transform. Consider the following series of values:\\
  \begin{equation*}
    3, 4, 3, 2, 1, 5, 6, 6
  \end{equation*}
  And the Haar weights:
 \begin{footnotesize}
 \begin{verbatim}
  s2 <- sqrt(2)
  s4 <- sqrt(4)
  HaarWt <- matrix(ncol=8, byrow=TRUE,
            c(1, 1, 1, 1, 1, 1, 1, 1,
              1, 1, 1, 1, -1, -1, -1, -1,
              s2, s2, -s2, -s2, 0, 0, 0, 0,
              0, 0, 0, 0, s2, s2, -s2, -s2,
              s4, -s4, 0, 0, 0, 0, 0, 0,
              0, 0, s4, -s4, 0, 0, 0, 0,
              0, 0, 0, 0, s4, -s4, 0, 0,
              0, 0, 0, 0, 0, 0, s4, -s4))
  x <- c(5,4,5,3,1,1,2,1)
  Haar_transform <- HaarWt%*%x
  Haar_transform


##          [,1]
## [1,] 22.000000
## [2,] 12.000000
## [3,] 1.414214
## [4,] -1.414214
## [5,] 2.000000
## [6,] 4.000000
## [7,] 0.000000
## [8,] 2.000000
\end{verbatim}
\end{footnotesize}

Essentially each row of the matrix is multiplied by x and the values added up (i.e. a dot product). Notice the large value for the 'large wavelength'
second wavelet value. This is because there are large values clumped in the first half of x and smaller values in the second half of x. This is typical
of wavelets, which have a frequency (how often they vary) and are local (based on only a small chunk of the data). Note that the first transform
isn't really worth looking at, as it's just an sum over everything.\\
\begin{enumerate}
  \item[a] Try to explain the pattern if you can guess it, or look it up?\\
  \solution 
  \vspace{.15in}




  \item[b] Repeat the Haar transform for the following dataset and interpret the wavelets:
\begin{footnotesize}
\begin{verbatim}
        x <- c(5,6,1,2,8,9,2,2) 
\end{verbatim}
\end{footnotesize}
You’ll notice that (ignoring the first one) two of the transform values are particularly large. What in the data are they detecting?\\
    \solution 

\end{enumerate}







\end{exercise}





\end{document}


















